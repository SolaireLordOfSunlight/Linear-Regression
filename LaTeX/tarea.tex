\documentclass{IEEEtran}

\usepackage{amsmath}
\usepackage[spanish]{babel}

\title{Analisis de Regresion con respecto a PPG frente a MPG en partidos de playoff de la NBA}
\author{Rudy Miranda}
\date{Abril, 2023}

\begin{document}
    \maketitle

    \section{Introducci\'on}
    Se busca confirmar si la relacion entre la cantidad de puntos anotados por un jugador de la NBA, frente a la cantidad de minutos jugados por partido tienen una relacion del tipo lineal.

    Los datos corresponden a los playoff (post-temporada) de la temporada 2021-22 de la NBA obtenidos de su sitio web oficial.

    Un punto a enfatizar es que las 79 unidades de  observacion son jugadores de la misma posicion, donde se toma en cuenta su promedio de puntos por partidos (PPG) y su promedio de minutos por partido (MPG).
    
    \section{Modelo Poblacional}
    
    \begin{equation}
        PPG = \beta_0 + \beta_1 MPG + \varepsilon        
    \end{equation}
    
    \section{Estimacion de Parametros}
    \section{Modelo Estimado}
    
    \begin{equation}
        PPG = \hat{\beta_0} + \hat{\beta_1} MPG + \varepsilon        
    \end{equation}
    
    \section{Analisis de Residuos}
    Una hipotesis que deben cumplir nuestro modelo es que $\varepsilon_i = (Y_i - \bar{Y}) \sim N(0, \sigma^2)$

    Aplicando distintos test de normalidad, tanto parametricos como no parametricos
    
    \begin{table}[h]
        \begin{center}
            \begin{tabular}{|c|c|}
                \hline Test & Valor P \\ \hline
                Jarque-Bera & 0.04469 \\ \hline
                Kolmogorov-Smirnov & 3.847e-05\\ \hline
                Shapiro-Wilk & 0.1174 \\ \hline
                Anderson-Darling & 0.1493 \\ \hline
            \end{tabular}
        \end{center}
    \end{table}

    Con un nivel de significancia de 0.05, podriamos considerar la distribucion como una normal.

    Ahora, al considerar un intervalo de confianza sobre la media con la hipotesis nula $H_0: \mu = 0$ obtenemos un valor $p = 1$, con lo que aceptamos $H_0$.

    Con ello se cumplen los dos supuestos de la distribucion de los residuos.

    \section{Conclusion}
    Es clara la relacion entre los minutos jugados y los puntos anotados, pero seria una buena desicion incluir mas variables a este modelo en vez de solo dejarlo en dos.
\end{document}
